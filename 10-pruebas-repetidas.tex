\chapter{Pruebas repetidas}

Un \emph{esquema de pruebas repetidas e independientes} es un experimento aleatorio, se repite una vez tras otra, sin que el resultado de cada realización pueda incluir en el resultado de los demás.

Una \emph{sucesión de variables aleatorias de Bernouilli} es una sucesión de variables aleatorias independientes $I_i$, cada una de las cuales vale 1 si se presenta el suceso $A$ en la prueba $i$-ésima y 0 en caso contrario.

\section{La distribución binomial y la aproximación de Poisson}

Si en cada realización de un experimento aleatorio el suceso $A$ tiene probabilidad $p$ de ocurrir, el repetir $n$ veces el experimento, el número de veces que se presenta el suceso $A$ es una variable aleatoria, $X_n$ con función de probabilidad.
\[
P\{X_n = k\} = \binom{n}{k} p^kq^{n-k} \qquad \text{para } k=0,1,2,\ldots,n
\]
donde $q = 1-p$. Se trata de una \emph{distribución binomial} $B(n,p)$.

Sus momentos más importantes son:
\[
E[X_n] = np \qquad \sigma^2(X_n) = np(1-p)
\]

\begin{theorem}[Teorema de Poisson]
    Si $n$ tiende a infinito y $p$ tienda a cero, de tal manera que el producto $np$ converge a una constante $\lambda$, se verifica
    \[
    \lim \binom{n}{k} p^k q^{n-k} = e^{-\lambda}\frac{\lambda^k}{k!}
    \]
    para cada valor fijo de $k$.
\end{theorem}

\section{La aproximación normal a la distribución binomial}

\begin{theorem}[Teorema de Moivre-Laplace]
    Cuando $n$ tiende a infinito, si $k$ tiende hacia infinito de manera que
    \[
    x_k=\frac{k-np}{\sqrt{npq}}
    \]
    se mantiene acotado en valor absoluto por alguna constante $A$, se cumple
    \[
    \binom{n}{k}p^k q^{n-k} \sim \frac{1}{\sqrt{2\pi npq}}e^{-\frac{x_k^2}{2}}
    \]
    en el sentido de que el cociente entre ambos miembros converge a 1. Más exactamente, se tiene
    \[
    \binom{n}{k}p^k q^{n-k} \sim \frac{1}{\sqrt{2\pi npq}}e^{-\frac{x_k^2}{2}}(1+\rho_n)
    \]
    donde $\frac{\sqrt{n}}{|\rho_n|}$ esta acotado, a partir de un $n$ en adelante, por una constante que solo depende de $A$ (y de $p$).
\end{theorem}