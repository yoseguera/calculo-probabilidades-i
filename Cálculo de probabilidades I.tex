% Options for packages loaded elsewhere
\PassOptionsToPackage{unicode}{hyperref}
\PassOptionsToPackage{hyphens}{url}
\documentclass[
]{article}
\usepackage{xcolor}
\usepackage{amsmath,amssymb}
\newtheorem{definition}{Definición}
\newtheorem{proposition}{Proposición}
\newtheorem{theorem}{Teorema}
\newtheorem{corollary}{Corolario}
\newtheorem{lemma}{Lema}
\renewcommand{\vec}[1]{\overrightarrow{#1}}
\newcommand{\R}{\mathbb{R}}
\newcommand{\K}{\mathbb{K}}
\newcommand{\prodes}[2]{\langle #1,#2 \rangle}
\newcommand{\norma}[1]{|| #1 ||}
\setcounter{secnumdepth}{-\maxdimen} % remove section numbering
\usepackage{iftex}
\ifPDFTeX
  \usepackage[T1]{fontenc}
  \usepackage[utf8]{inputenc}
  \usepackage{textcomp} % provide euro and other symbols
\else % if luatex or xetex
  \usepackage{unicode-math} % this also loads fontspec
  \defaultfontfeatures{Scale=MatchLowercase}
  \defaultfontfeatures[\rmfamily]{Ligatures=TeX,Scale=1}
\fi
\usepackage{lmodern}
\ifPDFTeX\else
  % xetex/luatex font selection
\fi
% Use upquote if available, for straight quotes in verbatim environments
\IfFileExists{upquote.sty}{\usepackage{upquote}}{}
\IfFileExists{microtype.sty}{% use microtype if available
  \usepackage[]{microtype}
  \UseMicrotypeSet[protrusion]{basicmath} % disable protrusion for tt fonts
}{}
\makeatletter
\@ifundefined{KOMAClassName}{% if non-KOMA class
  \IfFileExists{parskip.sty}{%
    \usepackage{parskip}
  }{% else
    \setlength{\parindent}{0pt}
    \setlength{\parskip}{6pt plus 2pt minus 1pt}}
}{% if KOMA class
  \KOMAoptions{parskip=half}}
\makeatother
\setlength{\emergencystretch}{3em} % prevent overfull lines
\providecommand{\tightlist}{%
  \setlength{\itemsep}{0pt}\setlength{\parskip}{0pt}}
\usepackage{bookmark}
\IfFileExists{xurl.sty}{\usepackage{xurl}}{} % add URL line breaks if available
\urlstyle{same}
\hypersetup{
  hidelinks,
  pdfcreator={LaTeX via pandoc}}

\author{}
\date{}

\begin{document}

\subsection{El modelo matemático de la
probabilidad}\label{el-modelo-matemuxe1tico-de-la-probabilidad}

\subsubsection{Espacio muestral y
sucesos}\label{espacio-muestral-y-sucesos}

Los resultados posibles constituyen un conjunto finito que se denomina
\textbf{espacio muestral} del fenómeno aleatorio en cuestos y se designa
genéricamente por \(\Omega\).

Los subconjuntos del espacio muestral \(\Omega\) se denominan
\textbf{sucesos} o \textbf{acontecimientos}, a los que puede dar lugar
el azar en el experimento considerado. Los subconjuntos con un único
elemento, representan uno sólo de los resultados posibles del fenómeno,
se llaman \textbf{sucesos simples}, mientras que los \textbf{sucesos
compuestos} se identifican con los subconjuntos que tienen más de un
elemento y son, por tanto, uniones de sucesos simples.

El espacio muestral \(\Omega\) es el \textbf{suceso seguro}, y el
subconjunto \(\emptyset\) es el \textbf{suceso imposible}.

Si la intersección de dos sucesos \(A\) y \(B\) es vacía
\(A \cap B = \emptyset\), se dice que \(A\) y \(B\) son
\textbf{incompatibles}*.

El \textbf{suceso contrario} es \(A^c = \Omega - A\).

\subsubsection{El concepto de
probabilidad}\label{el-concepto-de-probabilidad}

Sobre un espacio muestral finito \(\Omega\), una probabilidad es una
aplicación \[
P : P(\Omega) \longrightarrow [0,1]
\] que verifique: - \textbf{Aditividad}: Si
\(A, B \subset \Omega, A \cap B = \emptyset\), entonces
\(P(A \cup B) = P(A) + P(B)\). - \textbf{Normalización}:
\(P(\Omega) = 1\). El espacio de probabilidad finito es el par
\((\Omega,P)\). \#\#\# Primeras propiedades de la probabilidad * Si
\(A \subset B \subset \Omega\), entonces \(P(A) \leq P(B)\). *
\(P(B) = P(A) + P(B \cap A^c)\) * Si \(A \subset B \subset \Omega\),
entonces \(P(B - A) = P(B)- P(A)\). * Para cualquier
\(A \subset \Omega\), \(P(A^c) = 1 - P(A)\). * \(P(\emptyset) = 0\) *
\(P(B) = P(B \cap A) + P(B - A)\) *
\(P(A \cup B) = P(A) + P(B) + P(A \cap B)\). * Si
\(A_1, A_2,\ldots,A_n\) son sucesos disjuntos dos a dos, es decir que
verifiquen \(A_i \cap A_j = \emptyset\) para \(i \neq j\) se cumple *
\(P(A_1 \cup A_2 \cup \ldots \cup A_n) = P(A_1) + P(A_2) + \cdots + P(A_n)\)
* Si \(\Omega = \{\omega_1, \omega_2, \ldots, \omega_n \}\) es el
espacio muestral de probabilidad finito \((\Omega, P)\) y
\(A = \{\omega_1, \omega_2, \ldots, \omega_k\}\) es un suceso se cumple
*
\(P(A) = P(\{\omega_1\}) + P(\{\omega_2\}) + \cdots + P(\{\omega_k\})\)
\#\# Asignación de probabilidades \#\#\# Regla de Laplace La
probabilidad de un suceso \(A\) relativo a un fenómeno aleatorio es \[
P(A) = \frac{\text{número de casos favorables de A}}{\text{número de casos posibles}}
\] supuesto que todos los casos posibles son igualmente probables. \#\#
Las formulas de inclusión y exclusión \#\#\# Probabilidad de una unión
de sucesos Si \(A_1, A_2,\ldots,A_n\) son sucesos cualesquiera de un
espacio de probabilidad, se cumple \[
P(A_1 \cup A_2 \cup \cdots \cup A_n) = S_1 - S_2 + S_3- \cdots + (-1)^{n-2}S_{n-1}+(-1)^{n-1}S_n
\] donde \[
S_k = \sum_{1 \leq i_1 < i_2 < \cdots < i_k < n}P(A_{i_1} \cap A_{i_2} \cap \cdots \cap A_{i_k})
\] \#\#\# Probabilidad de que se realicen m sucesos En un espacio de
probabilidad, si \(A_1, A_2,\ldots,A_n\) son sucesos cualesquiera, la
probabilidad \(P_{[m]}\) de que ocurran exactamente \(m\) de ellos, es
\[
P_{[m]} = S_m - \binom{m+1}{m}S_{m+1} +\binom{m+2}{m}S_{m+2} - \cdots + (-1)^{n-m}\binom{n}{m}S_{n}
\]

\subsection{Extensiones del modelo
matemático}\label{extensiones-del-modelo-matemuxe1tico}

\subsubsection{Espacios muestrales
numerables}\label{espacios-muestrales-numerables}

Los conjuntos finitos son \textbf{discretos} y los infinitos numerables
son \textbf{numerables}.

Si \(\Omega\) es infinito numerable, el conjunto de sucesos,
\(P(\Omega)\), no es numerable, si no que su cardinal es aún mayor
(\(2^\mathbb{N}\)).

\subsubsection{Propiedades adicionales de la
probabilidad}\label{propiedades-adicionales-de-la-probabilidad}

Sobre un espacio muestral numerable \(\Omega\), una probabilidad es una
aplicación \[P:P(\Omega) \rightarrow [0,1]\] que verifique 1. Si \(A_n\)
es una sucesión de suceso tales que \(A_i \cap A_j = \emptyset\) para
\(i \neq j\),
entonces\[P\big(\bigcup_{n=1}^\infty A_n\big) = \sum_{n=1}^\infty P(A_n)\]
2. \(P(\Omega) = 1\) Si se cumplen ambas condiciones, \((\Omega,P)\)
constituye un espacio de probabilidad numerable.

En la construcciones de espacios de probabilidad no finitos, la regla de
Laplace para atribuir probabilidades resulta inservible directamente.

\paragraph{Modelo geométrico o de
Pascal}\label{modelo-geomuxe9trico-o-de-pascal}

\[\Omega = \mathbb{N} \qquad P(\{n\}) = p (1-p)^{n-1}\ \text{para cada }n \in \mathbb{N}\]
\#\#\# Modelos continuos No es posible que un modelo continuo asigne
probabilidad a todos los subconjuntos de \(\mathbb{R}\). Se toman en
cuenta los intervalos de \(\mathbb{R}\).

La probabilidad de un suceso \(A \subset \mathbb{R}\) no puede obtenerse
sumando la probabilidad de las puntos contenidos en \(A\).

Una función \(f: \mathbb{R} \rightarrow \mathbb{R}\) (con un número
finito de discontinuidades) se denomina \textbf{densidad de
probabilidad} si verifica 1. \(f(x) \geq 0\) para todo
\(x \in \mathbb{R}\) 2. \(\int_\mathbb{R} f(x)dx = 1\) El modelo
probabilístico asociado a la función de densidad \(f\) asigna
probabilidad \[P(x) = \int_I f(x)dx\] a cualquier intervalo \(I\) de
\(\mathbb{R}\).

Se puede aproximar mediante
\[P([x, x + \bigtriangleup x]) \simeq f(x)\bigtriangleup x\] \#\#\#\#
Distribución normal
\[\varphi(x) = \frac{1}{\sqrt{2\pi}}e^{\frac{-x^2}{2}}\] \#\#
Probabilidad condicionada \[
P(A|B) = \frac{\text{número de casos favorables de } A\text{ y }B}{\text{número de casos en los que ocurre }B}
\] Si \(A\) y \(B\) son sucesos de un cierto de espacio de probabilidad
y se cumple que \(P(B) > 0\), la probabilidad de \(A\) condicionada a
\(B\) es \[P(A|B) = \frac{P(A \cap B)}{P(B)}\] \[P(A)P(B|A)=P(B)P(A|B)\]
\#\#\# Propiedades 1. Si \(A_1,A_2,\ldots,A_n,\ldots\) son sucesos
disjuntos se
cumple\begin{eqnarray}P(A_1 \cup A_2 \cup \cdots \cup A_n \cup \cdots| B) \\&= \frac{P((A_1 \cap B) \cup (A_2 \cap B) \cup \cdots \cup (A_n \cap B)\cdots)}{P(B)} \\&= \frac{P((A_1 \cap B) + P(A_2 \cap B) + \cdots + P(A_n \cap B) + \cdots)}{P(B)} \\&= P(A_1|B) + P(A_2|B) + \cdots + P(A_n|B) + \cdots\end{eqnarray}
de forma que \(P(\cdot|B)\) es numeralmente aditiva. 2. Por otra
parte\[P(\Omega|B)=\frac{P(\Omega \cap B)}{P(B)} = 1\] y más
concretamente \[P(B|B) = 1\] \textgreater{[}!proposición{]} Formula de
las probabilidades totales \textgreater Si \(B_1,B_2,\ldots,B_n,\ldots\)
es una familia de sucesos de probabilidades positivas, disjuntos dos a
dos, tales que
\textgreater{}\[B_1 \cup B_2 \cup \cdots \cup B_n \cup \cdots = \Omega\]
\textgreater se verifica
\textgreater{}\[P(A) = P(B_1)P(A|B_1) + P(B_2)P(A|B_2) + \cdots+ P(B_n)P(A|B_n) + \cdots\]
\textgreater para cualquier suceso \(A \subset \Omega\).

\begin{quote}
{[}!proposición{]} Formula de Bayes Si \(B_1,B_2,\ldots,B_n,\ldots\) es
una familia de sucesos de probabilidades positivas, disjuntos dos a dos,
tales que \[B_1 \cup B_2 \cup \cdots \cup B_n \cup \cdots = \Omega\]
para cualquier suceso \(A\) de probabilidad positiva, de acuerdo con la
definición, se tiene
\[P(B_i|A) = \frac{P(B_i \cap A)}{P(A)} = \frac{P(B_i)P(A|B_i)}{P(A)}\]
lo cual, combinado con la fórmula de las probabilidades totales se puede
expresar
\[P(B_i|A) = \frac{P(B_i)P(A|B_i)}{P(B_1)P(A|B_1) + P(B_2)P(A|B_2) + \cdots+ P(B_n)P(A|B_n)}\]
\#\#\# El método recurrente Un situación en la que obtener la
probabilidad de un suceso dependa exclusivamente de lo ocurrido en la
etapa anterior se llama situación \textbf{markoviana}. \#\#
Independencia de sucesos \#\#\# Sucesos dependientes e independientes Si
un suceso \(A\) influye, favorablemente o desfavorablemente, en otro
\(B\), se dice que el segundo es \textbf{dependiente} del primero.
\end{quote}

En un fenómeno aleatorio, el suceso \(A\) se dice \textbf{independiente}
del suceso \(B\) (supuesto que \(B\) tiene probabilidad positiva) si
\[P(A|B) = P(A) \qquad \frac{P(A \cap B)}{P(B)} = P(A) \qquad P(A \cap B) = P(A)P(B)\]
Si \(A\) y \(B\) son independientes, también lo son \(A\) y \(B^c\).
\[P(A \cap B^c) = P(A)P(B^c)\] \#\#\# Espacios producto Dados dos
espacios de probabilidad \((\Omega_1,P_1)\) y \((\Omega_2, P_2)\), el
\textbf{espacio producto} \((\Omega,P)\) sería
\[\Omega = \Omega_1 \times \Omega_2 \qquad P=P(A \times B) = P_1(A)P_2(B)\]
Un \textbf{conjunto cilindro} de una sucesión de términos \(a_i\) de
\(\Omega^{\mathbb(N)}\) es aquella en la que los primeros \(n\) términos
deben cumplir una condición y el resto no esta sometido a ninguna
limitación. \#\#\# Independencia de varios sucesos En un espacio de
probabilidad, tres sucesos \(A_1, A_2, A_3\) son independientes si se
cumplen las condiciones 1. \(P(A_1 \cap A_2) = P(A_1)P(A_2)\) 2.
\(P(A_1 \cap A_3) = P(A_1)P(A_3)\) 3. \(P(A_2 \cap A_3) = P(A_2)P(A_3)\)
4. \(P(A_1 \cap A_2 \cap A_3) = P(A_1)P(A_2)P(A_3)\) Tres sucesos pueden
ser independientes dos a dos, sin ser independientes.

En un espacio de probabilidad, los sucesos \(\{A_i | i \in I\}\) se
dicen independientes si
\[P(A_{i_1} \cap A_{i_2} \cap \cdots \cap A_{i_k})= P(A_{i_1})P(A_{i_2}) \cdots P(A_{i_k})\]
cualquiera que sea \(k \in \mathbb{N}\) y cualesquiera que sea
\(i_1, i_2,\ldots, i_k \in I\). \#\#\# La independencia condicional Sea
\(A\), \(B\) y \(C\) tres sucesos en un espacio de probabilidad
\((\Omega,P)\). Si se verifica \[P(A \cap B|C) = P(A|C)P(B|C)\] se dice
que \(A\) y \(B\) son \textbf{independientes condicionalmente} a \(C\).

Los sucesos \(A\) y \(B\) pueden ser condicionalmente independientes,
tanto cuando ocurre \(C\) como cuando no ocurre, y no ser
independientes.

Dos sucesos \(A\) y \(B\) pueden ser independientes, per no ser
condicionalmente independientes a \(C\) ni a \(C^c\). \#\# Apéndice I.
Combinatoria \#\#\# Principios generales \textbf{Contar} es hallar el
\textbf{cardinal} de un conjunto. La \textbf{combinatoria} es el arte de
contar conjuntos sin hacer enumeraciones.

Si en una competición se inscriben \(n\) jugadores se juegan \(n-1\)
partidos.

El \textbf{procedimiento constructivo} consiste en recorrer mentalmente
los pasos a seguir para formar todos los elementos del conjunto,
anotando las alternativas que pueden elegirse en cada uno.

Si los conjuntos \(A_1, A_2, \ldots, A_n\) tienen
\(n_1, n_2,\ldots, n_n\) elementos respectivamente, el producto
cartesiano \(A_1 \times A_2 \times \cdots \times A_n\) tiene
\(n_1n_2\cdots n_n\) elementos.

No tiene ninguna relevancia donde se producen las restricciones, sólo
cuales son estas.

\subsubsection{Patrones más usuales}\label{patrones-muxe1s-usuales}

\paragraph{Ordenaciones}\label{ordenaciones}

\(n\) objetos distintos pueden ordenarse en fila de \(n!\) maneras
distintas.

Una colección de \(n\) objetos, clasificados en \(k\) grupos de objetos
idénticos entre sí, el primero con \(n_1\) objetos, el segundo con
\(n_2\),\ldots{} se pueden ordenar en fila de
\[\frac{n!}{n_1!n_2!\cdots n_k!}\] maneras distintas, si no se
consideran distintas las ordenaciones en las cuales dos objetos iguales
han permutado su posición. \#\#\#\# Subconjuntos ordenados Hay
\[n(n-1)(n-2)\cdots(n-r+1) = \frac{n!}{(n-r)!} = (n)_r\] subconjuntos
ordenados posibles de \(r\) elementos, que pueden extraerse de un
conjunto de \(n\) elementos. También conocidos como **variaciones sin
repetición de \(r\) elementos tomados entre \(n\).

El número de \textbf{variaciones con repetición} de \(r\) elementos
tomados entre \(n\) es \(n^r\). \#\#\#\# Subconjuntos El número de
subconjuntos distintos, con \(r\) elementos, que pueden extraerse de un
conjunto de \(n\) elementos es \[\binom{n}{r}=\frac{n!}{r!(n-r!)}\] o
\textbf{combinaciones} de \(r\) elementos tomados entre \(n\).

La diferencia entre variaciones y subconjuntos es si el orden de
aparición de los elementos es importante. \#\#\#\# Repartos Si hay que
repartir \(r\) objetos iguales en \(n\) grupos, existen
\[\binom{n+r-1}{r}\] repartos posibles. \#\#\# Identidades combinatorias
Una \textbf{identidad combinatoria} es una igualdad \(f(n)=g(n)\),
válida para cada \(n\) natural, en la cual \(f(n)\) y \(g(n)\) son
cantidades relacionadas con el cardinal de algún conjunto.
\[\binom{n}{k}=\binom{n}{n-k}\] \[2^n=\sum_{k=0}^n \binom{n}{k}\] \[
\sum_{i=0}^n \binom{n}{i}^2 = \binom{2n}{n}
\] \[
\sum_{k=0}^{m-1} \binom{n}{k}\binom{n-1}{m-k-1}=\binom{n+m-1}{n}
\] \[
\binom{n}{k}=\binom{n-1}{k}+\binom{n-1}{k-1}
\] \[
\binom{n}{k}=\sum_{j=0}^{k}\binom{n-j-1}{k-j}=\sum_{j=1}^{n-k+1}\binom{n-j}{k-1}
\] Si \(x \in \mathbb{R}\) y \(r >0\) \[
\binom{x}{r} = \frac{x(x-1)(x-2)\cdots(x-r+1)}{r!}
\] y si \(r < 0\) \[
\binom{x}{0} = 1 \qquad \binom{x}{r} = 0
\] \#\#\#\# Identidad inmediata \[
\binom{-x}{r} = (-1)^r\binom{x+r-1}{r}
\] \#\#\#\# Desarrollo de Taylor Si \(|t|<1\) y para cualquier
\(x \in \mathbb{R}\) \[
(1+t)^x=1 + \binom{x}{1}t + \binom{x}{2}t^2 + \cdots + \binom{x}{r}t^r+\cdots
\] y si \(|b| < |a|\) y cualquier \(x \in \mathbb{R}\) \[
(a+b)^x=a^x + \binom{x}{1}ba^{x-1} + \binom{x}{2}b^2a^{x-2} + \cdots + \binom{x}{r}b^ra^{a-r}+\cdots
\]

\[
(1-t)^n\sum_{r=0}^\infty \binom{n+r-1}{r}t^r=1
\] \[
(-1)^n\binom{-\frac{1}{2}}{n}=\binom{2n}{n}\frac{1}{2^{2n}}
\] \[
\sum_{n=0}^\infty \frac{2n}{n}\frac{t^n}{2^{2n}}=(1-t)^{\frac{-1}{2}}
\] y si \(n\) es un entero positivo \[
\binom{x+y}{n}=\binom{x}{0}\binom{y}{n}+\binom{x}{1}\binom{y}{n-1}+\binom{x}{2}\binom{y}{n-2}+\cdots+\binom{x}{n}\binom{y}{0}
\] \#\#\#\# Formula de Stirling \[
n! = \sqrt{2n}n^{n+\frac{1}{2}}e^{-n}
\] con más precisión \[
1 < \frac{n!}{\sqrt{2n}n^{n+\frac{1}{2}}e^{-n}} < e^{\frac{1}{8n}}
\]

\end{document}
