\chapter{Apéndice I. Combinatoria} 

\section{Principios generales} 

\textbf{Contar} es hallar el \textbf{cardinal} de un conjunto. La \textbf{combinatoria} es el arte de contar conjuntos sin hacer enumeraciones.

Si en una competición se inscriben \(n\) jugadores se juegan \(n-1\) partidos.

El \textbf{procedimiento constructivo} consiste en recorrer mentalmente los pasos a seguir para formar todos los elementos del conjunto,
anotando las alternativas que pueden elegirse en cada uno. 

Si los conjuntos \(A_1, A_2, \ldots, A_n\) tienen \(n_1, n_2,\ldots, n_n\) elementos respectivamente, el producto cartesiano \(A_1 \times A_2 \times \cdots \times A_n\) tiene
\(n_1n_2\cdots n_n\) elementos.

No tiene ninguna relevancia donde se producen las restricciones, solo cuales son estas.

\section{Patrones más usuales}
\label{patrones-muxe1s-usuales}

\subsection{Ordenaciones}
\label{ordenaciones}

\(n\) objetos distintos pueden ordenarse en fila de \(n!\) maneras distintas.

Una colección de \(n\) objetos, clasificados en \(k\) grupos de objetos idénticos entre sí, el primero con \(n_1\) objetos, el segundo con
\(n_2\),\ldots{} se pueden ordenar en fila de 
\[
\frac{n!}{n_1!n_2!\cdots n_k!}
\] 
maneras distintas, si no se consideran distintas las ordenaciones en las cuales dos objetos iguales han permutado su posición. 

\subsection{Subconjuntos ordenados} 

Hay \[n(n-1)(n-2)\cdots(n-r+1) = \frac{n!}{(n-r)!} = (n)_r\] subconjuntos ordenados posibles de \(r\) elementos, que pueden extraerse de un
conjunto de \(n\) elementos. También conocidos como \emph{variaciones sin repetición de \(r\) elementos tomados entre \(n\)}.

El número de \textbf{variaciones con repetición} de \(r\) elementos tomados entre \(n\) es \(n^r\). 

\subsection{Subconjuntos} 

El número de subconjuntos distintos, con \(r\) elementos, que pueden extraerse de un conjunto de \(n\) elementos es 
\[
\binom{n}{r}=\frac{n!}{r!(n-r!)}
\] 
o \textbf{combinaciones} de \(r\) elementos tomados entre \(n\).

La diferencia entre variaciones y subconjuntos es si el orden de aparición de los elementos es importante. 

\subsection{Repartos} 

Si hay que repartir \(r\) objetos iguales en \(n\) grupos, existen 
\[
\binom{n+r-1}{r}
\] 
repartos posibles. 

\section{Identidades combinatorias}

Una \textbf{identidad combinatoria} es una igualdad \(f(n)=g(n)\), válida para cada \(n\) natural, en la cual \(f(n)\) y \(g(n)\) son
cantidades relacionadas con el cardinal de algún conjunto. 
\[\binom{n}{k}=\binom{n}{n-k}\] \[2^n=\sum_{k=0}^n \binom{n}{k}\] \[
  \sum_{i=0}^n \binom{n}{i}^2 = \binom{2n}{n}
\] \[
  \sum_{k=0}^{m-1} \binom{n}{k}\binom{n-1}{m-k-1}=\binom{n+m-1}{n}
\] \[
  \binom{n}{k}=\binom{n-1}{k}+\binom{n-1}{k-1}
\] \[
  \binom{n}{k}=\sum_{j=0}^{k}\binom{n-j-1}{k-j}=\sum_{j=1}^{n-k+1}\binom{n-j}{k-1}
\] Si \(x \in \mathbb{R}\) y \(r >0\) \[
  \binom{x}{r} = \frac{x(x-1)(x-2)\cdots(x-r+1)}{r!}
\] y si \(r < 0\) \[
  \binom{x}{0} = 1 \qquad \binom{x}{r} = 0
\] \#\#\#\# Identidad inmediata \[
  \binom{-x}{r} = (-1)^r\binom{x+r-1}{r}
\] \#\#\#\# Desarrollo de Taylor Si \(|t|<1\) y para cualquier
\(x \in \mathbb{R}\) \[
  (1+t)^x=1 + \binom{x}{1}t + \binom{x}{2}t^2 + \cdots + \binom{x}{r}t^r+\cdots
\] y si \(|b| < |a|\) y cualquier \(x \in \mathbb{R}\) \[
  (a+b)^x=a^x + \binom{x}{1}ba^{x-1} + \binom{x}{2}b^2a^{x-2} + \cdots + \binom{x}{r}b^ra^{a-r}+\cdots
\]

\[
  (1-t)^n\sum_{r=0}^\infty \binom{n+r-1}{r}t^r=1
\] \[
  (-1)^n\binom{-\frac{1}{2}}{n}=\binom{2n}{n}\frac{1}{2^{2n}}
\] \[
  \sum_{n=0}^\infty \frac{2n}{n}\frac{t^n}{2^{2n}}=(1-t)^{\frac{-1}{2}}
\] y si \(n\) es un entero positivo \[
  \binom{x+y}{n}=\binom{x}{0}\binom{y}{n}+\binom{x}{1}\binom{y}{n-1}+\binom{x}{2}\binom{y}{n-2}+\cdots+\binom{x}{n}\binom{y}{0}
\] 

\section{Formula de Stirling} 
\[
  n! = \sqrt{2n}n^{n+\frac{1}{2}}e^{-n}
\] con más precisión \[
  1 < \frac{n!}{\sqrt{2n}n^{n+\frac{1}{2}}e^{-n}} < e^{\frac{1}{8n}}
\]

\chapter*{Desarrollo de series}

\begin{align}
	e^x &= \sum_{n=0}^\infty \frac{x^n}{n!}, \qquad x \in (-\infty,\infty) \\
	\sin x &= \sum_{n=0}^\infty (-1)^n \frac{x^{2n+1}}{(2n+1)!}, \qquad x \in (-\infty,\infty) \\ 
	\cos x &= \sum_{n=0}^\infty (-1)^n \frac{x^{2n}}{(2n)!}, \qquad x \in (-\infty,\infty) \\
	\frac{1}{1-x} &= \sum_{n=0}^\infty x^n, \qquad x \in (-1,1) \\
	\frac{1}{1+x} &= \sum_{n=0}^\infty (-1)^n x^n, \qquad x \in (-1,1) \\
	\log(1-x) &= \sum_{n=0}^\infty \frac{x^{n+1}}{n+1}, \qquad x \in (-1,1) \\
	\log(1+x) &= \sum_{n=0}^\infty (-1)^n \frac{x^{n+1}}{n+1}, \qquad x \in (-1,1) \\
	\log(\frac{1-x}{1+x}) &= 2\sum_{n=0}^\infty \frac{x^{2n+1}}{2n+1}, \qquad x \in (-1,1) \\
	\frac{1}{1+x*2} &= \sum_{n=0}^\infty (-1)^n x^{2n}, \qquad x \in (-1,1) \\
	\arctan(x) &= \sum_{n=0}^\infty (-1)^n \frac{x^{2n+1}}{2n+1}, \qquad x \in (-1,1) \\
	(1+x)^\alpha &= 1 + \sum_{n=1}^\infty \frac{\alpha(\alpha-1)\cdots(\alpha-n+1)}{n!}x^n, \qquad x \in (-1,1) \\
	\frac{1}{\sqrt{1+x}} &= 1 + \sum_{n=1}^\infty (-1)^n \frac{(2n-1)!!}{(2n)!!}x^n, \qquad x \in (-1,1) \\
	\frac{1}{\sqrt{1-x^2}} &= 1 +  \sum_{n=1}^\infty \frac{(2n-1)!!}{(2n)!!}x^{2n}, \qquad x \in (-1,1) \\
	\frac{1}{\sqrt{1+x^2}} &= 1 + \sum_{n=1}^\infty (-1)^n \frac{(2n-1)!!}{(2n)!!}x^{2n}, \qquad x \in (-1,1) \\
	\arcsin x &= x+ \sum_{n=1}^\infty  \frac{(2n-1)!!}{(2n)!!}\frac{x^{2n+1}}{2n+1}, \qquad x \in (-1,1) \\
	argsh x &= x+ \sum_{n=1}^\infty  (-1)^n \frac{(2n-1)!!}{(2n)!!}\frac{x^{2n+1}}{2n+1}, \qquad x \in (-1,1) 
\end{align}

\end{document}
