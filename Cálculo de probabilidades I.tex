\documentclass[11pt,a4paper,twoside]{memoir}

% ---------------------------------------------------------
% PAQUETES BÁSICOS
% ---------------------------------------------------------
\usepackage[utf8]{inputenc}
\usepackage[T1]{fontenc}
\usepackage{lmodern} % Puedes cambiar por Garamond, Libertinus, etc.
\usepackage[spanish]{babel}
\usepackage{microtype} % Mejora tipográfica
\usepackage{graphicx}
\usepackage{amsmath, amssymb}
\usepackage{hyperref}

% ---------------------------------------------------------
% CONFIGURACIÓN DE PÁGINA Y MÁRGENES
% ---------------------------------------------------------
\setlrmarginsandblock{3.5cm}{3.5cm}{*}
\setulmarginsandblock{3cm}{3cm}{*}
\checkandfixthelayout

% ---------------------------------------------------------
% ESTILO DE CAPÍTULOS (ejemplo clásico)
% ---------------------------------------------------------
\chapterstyle{hangnum}

% Otros estilos disponibles:
% demo, default, section, companion, article, bianchi, veelo, ell, dash, madsen, pedersen, chappell, bringhurst...

% ---------------------------------------------------------
% CABECERAS Y PIES DE PÁGINA
% ---------------------------------------------------------
\makepagestyle{mystyle}
\makeevenhead{mystyle}{}{}{\thepage}
\makeoddhead{mystyle}{\thepage}{}{}
\makeevenfoot{mystyle}{}{}{}
\makeoddfoot{mystyle}{}{}{}
\pagestyle{mystyle}

% Símbolos mátemáticos
\newcommand{\e}{\operatorname{e}}
\newcommand{\ii}{\operatorname{i}}
\newcommand{\Z}{\mathbb{Z}}
\newcommand{\N}{\mathbb{N}}
\newcommand{\K}{\mathbb{K}}
\newcommand{\Q}{\mathbb{Q}}
\renewcommand{\emph}[1]{\textbf{#1}}
\newcommand{\C}{\mathbb{C}}
\newcommand{\R}{\mathbb{R}}
%\renewcommand{\mod}{aaa}
\newcommand{\Mod}[1]{\ \operatorname{MOD}\ #1}
\newcommand{\mcd}[1]{\operatorname{mcd}(#1)}
\newcommand{\mcm}[1]{\operatorname{mcm}(#1)}
\newcommand{\card}[1]{\operatorname{card} #1}
\newcommand{\anipol}[2]{#1[#2_1 \cdots #2_n]}
\newcommand{\anipolg}{\anipol{A}{X}}
\newcommand{\anipolgk}{\anipol{K}{X}}
\newcommand{\polinomio}[3]{\sum_{#2} #1_{#2} #3_1^{#2_1} \cdots #3_n^{#2_n}}
\newcommand{\polinomiog}{\polinomio{a}{v}{X}}
\newcommand{\diag}{\operatorname{diag}}
%\newcommand{\rg}[1]{\operatorname{rg}\left(#1\right)}
\newcommand{\rg}[1]{\operatorname{rg}\left(#1\right)}
\DeclareMathOperator{\adj}{Adj}
\let\ker\relax
\DeclareMathOperator{\ker}{ker}
\DeclareMathOperator{\im}{im}
\DeclareMathOperator{\SP}{sp}
\DeclareMathOperator{\Id}{Id}
\DeclareMathOperator{\GrTr}{gr.trans.}
\newcommand{\grado}{\partial}
\newcommand{\fix}{\operatorname{Fix}}
\newcommand{\fixm}{\operatorname{Fix}^-}
\newcommand{\id}{\operatorname{Id}}
\newcommand{\Grad}{\operatorname{grad}}
\newcommand{\Gen}[1]{\langle#1\rangle}
\newcommand{\biy}{\operatorname{Biy}}
\newcommand{\Rel}{\mathcal{R}}
\newcommand{\cont}{\mathbf{c}}
\renewcommand{\vec}[1]{\overrightarrow{#1}}
\newcommand{\prodes}[2]{\langle #1,#2 \rangle}
\newcommand{\norma}[1]{|| #1 ||}

\newtheorem{definition}{Definici\'{o}n}[chapter]
\newtheorem{theorem}[definition]{Teorema}
\newtheorem{proposition}[definition]{Proposici\'{o}n}
\newtheorem{corollary}[definition]{Corolario}
\newtheorem{lemma}[definition]{Lema}
\newtheorem{construccion}[definition]{Construcci\'{o}n}
\newtheorem{algoritmo}[definition]{Algoritmo}

% ---------------------------------------------------------
% INICIO DEL DOCUMENTO
% ---------------------------------------------------------
\title{Cálculo de probabilidades I}
\date{\today}

\begin{document}

\maketitle

% Dedicatoria opcional
\cleardoublepage
\thispagestyle{empty}
\vspace*{5cm}
\begin{center}
  \emph{A quien corresponda.}
\end{center}
\cleardoublepage


\tableofcontents

\chapter{El modelo matemático de la probabilidad}
\label{el-modelo-matemuxe1tico-de-la-probabilidad}

\section{Espacio muestral y sucesos}
\label{espacio-muestral-y-sucesos}

Los resultados posibles constituyen un conjunto finito que se denomina \textbf{espacio muestral} del fenómeno aleatorio en cuestos y se designa
genéricamente por \(\Omega\).

Los subconjuntos del espacio muestral \(\Omega\) se denominan \textbf{sucesos} o \textbf{acontecimientos}, a los que puede dar lugar
el azar en el experimento considerado. Los subconjuntos con un único elemento, representan uno sólo de los resultados posibles del fenómeno,
se llaman \textbf{sucesos simples}, mientras que los \textbf{sucesos compuestos} se identifican con los subconjuntos que tienen más de un
elemento y son, por tanto, uniones de sucesos simples.

El espacio muestral \(\Omega\) es el \textbf{suceso seguro}, y el subconjunto \(\emptyset\) es el \textbf{suceso imposible}.

Si la intersección de dos sucesos \(A\) y \(B\) es vacía \(A \cap B = \emptyset\), se dice que \(A\) y \(B\) son
\textbf{incompatibles}.

El \textbf{suceso contrario} es \(A^c = \Omega - A\).

\section{El concepto de probabilidad}
\label{el-concepto-de-probabilidad}

Sobre un espacio muestral finito \(\Omega\), una probabilidad es una aplicación 
\[
  P : P(\Omega) \longrightarrow [0,1]
\] 
que verifique: 
\begin{enumerate}
  \item \textbf{Aditividad}: Si \(A, B \subset \Omega, A \cap B = \emptyset\), entonces \(P(A \cup B) = P(A) + P(B)\). 
  \item \textbf{Normalización}: \(P(\Omega) = 1\).
\end{enumerate}

El espacio de probabilidad finito es el par \((\Omega,P)\). 

\section{Primeras propiedades de la probabilidad} 

\begin{itemize}
  \item Si \(A \subset B \subset \Omega\), entonces \(P(A) \leq P(B)\).
  \item \(P(B) = P(A) + P(B \cap A^c)\) * Si \(A \subset B \subset \Omega\), entonces \(P(B - A) = P(B)- P(A)\). 
  \item Para cualquier \(A \subset \Omega\), \(P(A^c) = 1 - P(A)\). 
  \item \(P(\emptyset) = 0\)
  \item \(P(B) = P(B \cap A) + P(B - A)\)
  \item \(P(A \cup B) = P(A) + P(B) + P(A \cap B)\). 
  \item Si \(A_1, A_2,\ldots,A_n\) son sucesos disjuntos dos a dos, es decir que verifiquen \(A_i \cap A_j = \emptyset\) para \(i \neq j\) se cumple
  \[
  P(A_1 \cup A_2 \cup \ldots \cup A_n) = P(A_1) + P(A_2) + \cdots + P(A_n)
  \]
  \item Si \(\Omega = \{\omega_1, \omega_2, \ldots, \omega_n \}\) es el espacio muestral de probabilidad finito \((\Omega, P)\) y \(A = \{\omega_1, \omega_2, \ldots, \omega_k\}\) es un suceso se cumple
  \[
  P(A) = P(\{\omega_1\}) + P(\{\omega_2\}) + \cdots + P(\{\omega_k\})
  \]
\end{itemize}


\chapter{Asignación de probabilidades}

\section{Regla de Laplace} 

La probabilidad de un suceso \(A\) relativo a un fenómeno aleatorio es 
\[
  P(A) = \frac{\text{número de casos favorables de A}}{\text{número de casos posibles}}
\] 
supuesto que todos los casos posibles son igualmente probables.

\chapter{Las formulas de inclusión y exclusión}

\subsection{Probabilidad de una unión de sucesos}

Si \(A_1, A_2,\ldots,A_n\) son sucesos cualesquiera de un espacio de probabilidad, se cumple 
\[
  P(A_1 \cup A_2 \cup \cdots \cup A_n) = S_1 - S_2 + S_3- \cdots + (-1)^{n-2}S_{n-1}+(-1)^{n-1}S_n
\] 
donde 
\[
  S_k = \sum_{1 \leq i_1 < i_2 < \cdots < i_k < n}P(A_{i_1} \cap A_{i_2} \cap \cdots \cap A_{i_k})
\] 

\subsection{Probabilidad de que se realicen m sucesos }

En un espacio de probabilidad, si \(A_1, A_2,\ldots,A_n\) son sucesos cualesquiera, la probabilidad \(P_{[m]}\) de que ocurran exactamente \(m\) de ellos, es
\[
  P_{[m]} = S_m - \binom{m+1}{m}S_{m+1} +\binom{m+2}{m}S_{m+2} - \cdots + (-1)^{n-m}\binom{n}{m}S_{n}
\]

\chapter{Extensiones del modelo matemático}
\label{extensiones-del-modelo-matemuxe1tico}

\section{Espacios muestrales numerables}
\label{espacios-muestrales-numerables}

Los conjuntos finitos son \textbf{discretos} y los infinitos numerables son \textbf{numerables}.

Si \(\Omega\) es infinito numerable, el conjunto de sucesos, \(P(\Omega)\), no es numerable, si no que su cardinal es aún mayor (\(2^\mathbb{N}\)).

\section{Propiedades adicionales de la probabilidad}
\label{propiedades-adicionales-de-la-probabilidad}

Sobre un espacio muestral numerable \(\Omega\), una probabilidad es una aplicación \[P:P(\Omega) \rightarrow [0,1]\] que verifique:
\begin{enumerate}
  \item Si \(A_n\) es una sucesión de suceso tales que \(A_i \cap A_j = \emptyset\) para \(i \neq j\), entonces
  \[
  P\left(\bigcup_{n=1}^\infty A_n\right) = \sum_{n=1}^\infty P(A_n)
  \]
  \item \(P(\Omega) = 1\) Si se cumplen ambas condiciones, \((\Omega,P)\) constituye un espacio de probabilidad numerable.
\end{enumerate}


En las construcciones de espacios de probabilidad no finitos, la regla de Laplace para atribuir probabilidades resulta inservible directamente.

\subsection{Modelo geométrico o de Pascal}
\label{modelo-geomuxe9trico-o-de-pascal}

\[
\Omega = \mathbb{N} \qquad P(\{n\}) = p (1-p)^{n-1}\ \text{para cada }n \in \mathbb{N}
\]

\section{Modelos continuos} 

No es posible que un modelo continuo asigne probabilidad a todos los subconjuntos de \(\mathbb{R}\). Se toman en cuenta los intervalos de \(\mathbb{R}\).

La probabilidad de un suceso \(A \subset \mathbb{R}\) no puede obtenerse sumando la probabilidad de los puntos contenidos en \(A\).

Una función \(f: \mathbb{R} \rightarrow \mathbb{R}\) (con un número finito de discontinuidades) se denomina \textbf{densidad de   probabilidad} si verifica 1. \(f(x) \geq 0\) para todo
\(x \in \mathbb{R}\) 2. \(\int_\mathbb{R} f(x)dx = 1\) El modelo probabilístico asociado a la función de densidad \(f\) asigna probabilidad \[P(x) = \int_I f(x)dx\] a cualquier intervalo \(I\) de
\(\mathbb{R}\).

Se puede aproximar mediante 
\[
P([x, x + \bigtriangleup x]) \simeq f(x)\bigtriangleup x
\] 

\subsection{Distribución normal}

\[
\varphi(x) = \frac{1}{\sqrt{2\pi}}e^{\frac{-x^2}{2}}
\]


\chapter{Probabilidad condicionada}

\[
  P(A|B) = \frac{\text{ número de casos favorables de } A\text{ y }B}{\text{ número de casos en los que ocurre }B}
\] 

Si \(A\) y \(B\) son sucesos de un cierto de espacio de probabilidad y se cumple que \(P(B) > 0\), la probabilidad de \(A\) condicionada a
\(B\) es 
\[
P(A|B) = \frac{P(A \cap B)}{P(B)}\] \[P(A)P(B|A)=P(B)P(A|B)
\] 

\section{Propiedades} 

\begin{enumerate}
  \item Si \(A_1,A_2,\ldots,A_n,\ldots\) son sucesos disjuntos se cumple
  \begin{align*}
    P(A_1 \cup A_2 \cup \cdots \cup A_n &\cup \cdots| B) \\&= \frac{P((A_1 \cap B) \cup (A_2 \cap B) \cup \cdots \cup (A_n \cap B)\cdots)}{P(B)} \\&= \frac{P((A_1 \cap B) + P(A_2 \cap B) + \cdots + P(A_n \cap B) + \cdots)}{P(B)} \\&= P(A_1|B) + P(A_2|B) + \cdots + P(A_n|B) + \cdots
  \end{align*}
  de forma que \(P(\cdot|B)\) es numeralmente aditiva. 
  \item Por otra parte
  \[
    P(\Omega|B)=\frac{P(\Omega \cap B)}{P(B)} = 1
  \] 
  y más concretamente 
  \[
  P(B|B) = 1
  \] 
\end{enumerate}

\begin{proposition}[Formula de las probabilidades totales]
Si \(B_1,B_2,\ldots,B_n,\ldots\) es una familia de sucesos de probabilidades positivas, disjuntos dos a dos, tales que
\[
B_1 \cup B_2 \cup \cdots \cup B_n \cup \cdots = \Omega
\]
se verifica
\[
P(A) = P(B_1)P(A|B_1) + P(B_2)P(A|B_2) + \cdots+ P(B_n)P(A|B_n) + \cdots
\]
para cualquier suceso \(A \subset \Omega\).  
\end{proposition}


\begin{proposition}[Formula de Bayes]
  Si \(B_1,B_2,\ldots,B_n,\ldots\) es una familia de sucesos de probabilidades positivas, disjuntos dos a dos, tales que 
  \[
  B_1 \cup B_2 \cup \cdots \cup B_n \cup \cdots = \Omega
  \]
  para cualquier suceso \(A\) de probabilidad positiva, de acuerdo con la definición, se tiene
  \[
  P(B_i|A) = \frac{P(B_i \cap A)}{P(A)} = \frac{P(B_i)P(A|B_i)}{P(A)}
  \]
  lo cual, combinado con la fórmula de las probabilidades totales se puede expresar
  \[
  P(B_i|A) = \frac{P(B_i)P(A|B_i)}{P(B_1)P(A|B_1) + P(B_2)P(A|B_2) + \cdots+ P(B_n)P(A|B_n)}
  \]  
\end{proposition}

\section{El método recurrente} 

Una situación en la que obtener la probabilidad de un suceso dependa exclusivamente de lo ocurrido en la etapa anterior se llama situación \textbf{markoviana}.

\chapter{Independencia de  sucesos}

\section{Sucesos dependientes e independientes} 

Si un suceso \(A\) influye, favorablemente o desfavorablemente, en otro \(B\), se dice que el segundo es \textbf{dependiente} del primero.

En un fenómeno aleatorio, el suceso \(A\) se dice \textbf{independiente} del suceso \(B\) (supuesto que \(B\) tiene probabilidad positiva) si
\[
P(A|B) = P(A) \qquad \frac{P(A \cap B)}{P(B)} = P(A) \qquad P(A \cap B) = P(A)P(B)
\]

Si \(A\) y \(B\) son independientes, también lo son \(A\) y \(B^c\).
\[
P(A \cap B^c) = P(A)P(B^c)
\] 

\section{Espacios producto} 

Dados dos espacios de probabilidad \((\Omega_1,P_1)\) y \((\Omega_2, P_2)\), el \textbf{espacio producto} \((\Omega,P)\) sería 
\[
\Omega = \Omega_1 \times \Omega_2 \qquad P=P(A \times B) = P_1(A)P_2(B)
\]

Un \textbf{conjunto cilindro} de una sucesión de términos \(a_i\) de \(\Omega^{\mathbb(N)}\) es aquella en la que los primeros \(n\) términos
deben cumplir una condición y el resto no esta sometido a ninguna limitación. 

\section{Independencia de varios sucesos} 

En un espacio de probabilidad, tres sucesos \(A_1, A_2, A_3\) son independientes si se cumplen las condiciones
\begin{enumerate}
  \item \(P(A_1 \cap A_2) = P(A_1)P(A_2)\)
  \item \(P(A_1 \cap A_3) = P(A_1)P(A_3)\) 
  \item \(P(A_2 \cap A_3) = P(A_2)P(A_3)\)
  \item \(P(A_1 \cap A_2 \cap A_3) = P(A_1)P(A_2)P(A_3)\)
\end{enumerate} 

Tres sucesos pueden ser independientes dos a dos, sin ser independientes.

En un espacio de probabilidad, los sucesos \(\{A_i | i \in I\}\) se dicen independientes si 
\[
P(A_{i_1} \cap A_{i_2} \cap \cdots \cap A_{i_k})= P(A_{i_1})P(A_{i_2}) \cdots P(A_{i_k})
\]
cualquiera que sea \(k \in \mathbb{N}\) y cualesquiera que sea \(i_1, i_2,\ldots, i_k \in I\). 


\section{La independencia condicional} 

Sea \(A\), \(B\) y \(C\) tres sucesos en un espacio de probabilidad \((\Omega,P)\). Si se verifica 
\[
P(A \cap B|C) = P(A|C)P(B|C)
\] se dice que \(A\) y \(B\) son \textbf{independientes condicionalmente} a \(C\).

Los sucesos \(A\) y \(B\) pueden ser condicionalmente independientes, tanto cuando ocurre \(C\) como cuando no ocurre, y no ser independientes.

Dos sucesos \(A\) y \(B\) pueden ser independientes, per no ser condicionalmente independientes a \(C\) ni a \(C^c\). 

\chapter{Variables aleatorias}

\section{El concepto de variable aleatoria}

En un espacio de probabilidad discreto \((\Omega,P)\), se denomina \emph{variable aleatoria}, a cualquier función
\[
X: \Omega \rightarrow \R
\]

Si la variable aleatoria \(X\) estuviese definida en un espacio muestral no numerable no habría ninguna garantías que conjuntos del tipo \(X^{-1}(a,b]\) fuesen sucesos.

\section{Distribución de una variable aleatoria}

La colección de probabilidades \(=\{X \in B\}\), correspondientes a cada subconjunto \(B\) de $\R$, se denomina la \emph{distribución de la variable aleatoria $X$}.

Una variable aleatoria $X$ es \emph{discreta} si toma a lo sumo un número numerable de valores
\[
X(\Omega) = \{x_1, x_2, x_2, \ldots, x_n\}
\]
La función que a cada $x_k$ asigna el valores
\[
p_k =  P\{X = x_k\}
\]
se denomina \emph{función de probabilidad de $X$}.

Naturalmente debe ser
\[
\sum_{k=1}^\infty p_k = 1
\]

Una urna en la que se introducen tarjetas que llevan anotado uno de los números $x_i$. Si la proporción de tarjetas con el número $x_i$ es $p_i$, el número obtenido al elegir una tarjeta al azar, es una variable aleatoria con la función de probabilidad requerida\footnote{El mecanismo tropezará con dificultades si alguna de las $p_i$ es irracional, por la imposibilidad de que, con un número finito de tarjetas, la proporción de algún subconjunto de ellas sea irracional. Una alternativa es dividir $\Omega = [0,1]$ en segmentos de longitud $p_1,p_2,\ldots,p_i,\ldots$ y escoger un punto con distribución uniforme en $[0,1]$.}.

\section{Variables aleatorias simultaneas}

La igualdad entre variables aleatorias exige que ambas estén definidas en el mismo espacio de probabilidad; en cambio la igualdad de sus distribuciones puede ocurrir entre variables definidas en espacios distintos.

Dos variables aleatorias tendrán una \emph{relación funcional} cuando una vez elegegida una de ellas, la segunda queda determinada sin ninguna aleatoriedad.

Si \(X_1, X_2\) son variables aleatorias discretas definidas en el mismo espacio de probabilidad, la distribución conjunta de \((X_1,X_2)\) es la función que asigna a cada $B \subset \R^2$ la probabilidad
\[
P\{(X_1,X_2) \in B \}
\]

En realidad, la distribución conjunto de $(X_1,X_2)$ se caracteriza por la función de probabilidad conjunta que hace corresponder
\[
P\{X_1=x_1, X_2=x_2\}
\]
a cada par $(x_1,x_2)$ de valores posibles de $(X_1,X_2)$ (que siempre será un subconjunto de $X_1(\Omega) \times X_2(\Omega)$).

Las funciones de probabilidad $P\{X_1 = x_2\}$ y $P\{X_2 = x_2\}$ se denominan \emph{funciones de probabilidad marginales} de $X_2$ y $X_2$ respectivamente. Claramente
\begin{align*}
  P\{X_1 = x_1\} &= \sum_{x_2 \in X_2(\Omega)} P\{X_1 = x_1, X_2 = x_2\} \\
  P\{X_1 = x_1\} &= \sum_{x_1 \in X_1(\Omega)} P\{X_1 = x_1, X_2 = x_2\}
\end{align*}

La función de Probabilidad
\[
P\{X_2 = x_2 | X_1 = x_1\} = \frac{P\{X_1 = x_1, X_2 = x_2\}}{P\{X_1 = x_1\}}
\]
donde $x_1$ es fijo y $x_2$ variable, corresponde a la \emph{distribución de $X_2$ condicionada por $X_1 = x_1$}. De igual forma $P\{X_1 = x_1 | X_2 = x_2\}$ (con $x_2$ fijo y $x_1$ variable) es la función de probabilidad de la distribución de $X_1$ condicionada por $X_2 = x_2$.

\section{Variables aleatorias independientes}

Dos variables aleatorias $X_1$ y $X_2$, definidas en el mismo espacio probabilidad, se denominas \emph{independientes} si se verifica
\[
P\{X_1 = x_1, X_2 = x_2\} = P\{X_1 = x_1\}P\{X_2 = x_2\}
\]
cualquiera que sean $x_1$ y $x_2$ entre los valores posibles de las variables.

Las variables aleatorias discretas $X_1,X_2,\ldots,X_r$, definidas en el mismo espacio de probabilidad, se denominan \emph{independientes} si
\[
P\{X_1 = x_1, X_2 = x_2,\ldots,X_r=x_r\} = P\{X_1 = x_1\}P\{X_2 = x_2\} \cdots P\{X_r = x_r\}
\]
cualquiera que sean $x_1,x_2,\ldots,x_r$ dentro de los valores posibles de $X_1,X_2,\ldots,X_r$ respectivamente.

\begin{lemma}
  Si $X_1,X_2,\ldots,X_r$ son variables aleatorias discretas e independientes y $C_1,C_2,\ldots,C_r$ sus subconjuntos cualesquiera de sus conjuntos de valores posibles, se verifica
  \[
  P\{X_1 \in C_1, X_2 \in C_2,\ldots,X_r \in C_r\} = P\{X_1 \in C_1\}P\{X_2 \in C_2\} \cdots P\{X_r \in C_r\}
  \]
  En consecuencia, los sucesos $\{X_1 \in C_1\}, \{X_2 \in C_2\},\ldots,\{X_r \in C_r\}$ son independientes.
\end{lemma}

\chapter{Esperanza matemática}

Si $X$ es una variable aleatoria definida en un espacio de probabilidad discreto $(\Omega,P)$, el \emph{valor esperado}\footnote{Antiguamente, se conocía como \emph{esperanza moral}}, \emph{esperanza matemática} o \emph{media} de $X$ es
\[
E[X] = \sum_{\omega \in \Omega} X(\omega)P(\omega)
\]
en el supuesto que
\[
\sum_{\omega|X(\omega)>0} X(\omega)P(\omega) < \infty \text{ o } \sum_{\omega|X(\omega)<0} -X(\omega)P(\omega) < \infty
\]
Si las dos series anteriores divergen, se dice que $E[X]$ no existe mientras que, si la primera diverge y la segunda converge, se toma $E[X]=+\infty$ y, si es al revés, $E[X]=-\infty$.

Si el conjunto de valores posibles de una variable aleatoria $X$ es
\[X(\Omega)=\{x_1,x_2,\ldots,x_n\}\]
su \emph{valor esperado}, si existe, se expresa
\[
E[X] = \sum_k x_k P\{X=x_k\} = \sum_k x_kp_k
\]
donde $p_k$ es la función de probabilidad de $X$.

\section{Propiedades de la esperanza matemática}

\begin{enumerate}
    \item Cualquier constante $c \in \R$ puede considerarse una variable aleatoria, basta definir $X(\omega) = c$ para todo $\omega \in \Omega$. Su distribución se denomina \emph{distribución causal en $c$}.
    \[E[c]=c\]
    \item \emph{Linealidad}: Sean $X_1$ y $X_2$ variables aleatorias discretas en el mismo espacio de probabilidad, cuyas medias son finitas. Si $c_1,c_2 \in \R$
    \[E[c_1X_1 + c_2X_2]=c_1E[X_1]+c_2E[X_2]\] 
    \item Sea $X$ una variable aleatoria no negativa, es decir $X(\omega) \geq 0$ para todo $\omega \in \Omega$. Entonces
    \begin{align*}
        E[X] &\geq 0 \\
        X \geq Y &\Rightarrow E[X] \geq E[Y]
    \end{align*}
    \item Si $I_A$ es la función indicatriz de un suceso $A$, se tiene
    \[E[I_A]=P(A)\]
    \item si $X$ es una variable aleatoria que toma solo valores enteros no negativos, se cumple
    \[E[X] = \sum_{m=1}^{\infty}P\{X \geq m\}\]
    o lo que es lo mismo
    \[E[X] = \sum_{m=0}^{\infty}P\{X > m\}\]
    Por consiguiente, para una variable aleatoria $X$ con valores enteros
    \[E[|X|] = \sum_{m=1}^{\infty}P\{|X| \geq m\}\]
    Las probabilidades $P\{|X| \geq m\}$ se denominan \emph{colas de la distribución}, porque dan la probabilidad de que $X$ esté fuera de la región central $(-m,m)$.
    Si $X=f(X)$ es una variable aleatoria función no lineal de otra, suele ser
    \[E[f(X)] \neq f(E[X])\]
    \item Si $X$, $Y$ son variables aleatorias independientes, con esperanza finita, se verifica
    \[E[XY] = E[X]E[Y]\]
\end{enumerate} 

\section{Esperanza condicionada y métodos recurrentes}

El que haya ocurrido un suceso $B$, de probabilidad $P(B) > 0$, da lugar la \emph{esperanza matemática condicionada por $B$}
\[
E[X|B] = \sum_{\omega \in \Omega} X(\omega)P\{\omega | B\} = \sum_k x_k P\{X=x_k|B\}
\]

\chapter{Análisis descriptivo de las distribuciones de probabilidad}

\section{Momentos de una distribución}

\subsection{Respecto al origen}

El \emph{momento de orden $r$ respecto del origen} de una variable aleatoria $X$, o de su distribución de probabilidad, es la esperanza matemática de $X^r$:
\[
\alpha_r = E[X^r] = \sum_{\omega \in \Omega} X^r(\omega)P\{\Omega\} = \sum_k x_k^r p_k
\]
que, cuando existe, se suele designar por $\alpha_r$.

Salvo cuando $X$ es positiva, no es conveniente considerar más que momentos de orden entero.

Cuando el momento de orden $r$ es finito, también son finitos los momentos de orden inferior a $r$.

\subsection{Respecto a la media o momento central} 

El \emph{momento de orden $r > 1$ respecto a la media o momento central de orden $r$} de una variable aleatoria $X$, o de su distribución, es la esperanza matemática de $(X-E[X])^r$. Por tanto, si se le designa por $\mu_r$, es
\[
\mu_r = E[(X-E[X])^r] = \sum_k (x_k -\alpha_1)^r p_k
\]

Los momentos centrales respecto al origen se relacionan
\[
\mu_r = \sum_{i=0}^r (-1)^i \binom{r}{i}\alpha_1^i\alpha_{r-1}
\]
siendo $\alpha_0 = 1$.

Por ejemplo,
\begin{align*}
    \mu_2 &= \alpha_2 - \alpha_1^2 \\
    \mu_3 &= \alpha_3 - 3\alpha_1\alpha_2 + 2\alpha_1^3 \\
    \mu_4 &= \alpha_4 - 4 \alpha_1\alpha_3 + 6\alpha_1^2\alpha_2 - 3\alpha_1^4
\end{align*}

El momento central de segundo orden, $\mu_2$ es el equivalente al momento de inercia y se le conoce como \emph{varianza de la distribución}, y se le designa como $V(X)$ o $\sigma^2(X)$.
\[
V(X) = \sigma^2 = E[(X-E[X])^2] = \sum_k (x_k - \alpha_1)^2p_k = E[X^2] - E[X]^2
\]
Como $\sigma^2 \geq 0$ siempre, se tiene que
\[
E[X^2] \geq E[X]^2
\]

La \emph{desviación típica de la distribución}, $\sigma$, es la raíz cuadrada de la varianza ($\sigma^2$) y se interpreta como la dispersión de la distribución alrededor de la media.

A veces, se utiliza \emph{coeficiente de varianza de la variable $X$}
\[
\frac{\sigma(X)}{E[X]}
\]

Si la dispersión se calcula a partir de un punto genérico $a$, mediante $E[(X-a)^2]$. Esta se minimiza si calcula alrededor de la media, $a=E[X]$.

\begin{proposition}[Desigualdad de Tchebychev]
    Cualquiera que sea la distribución de probabilidad de una variable aleatoria $X$, se verifica
    \[
    P\{|X-E[X]| > k\sigma\} \leq \frac{1}{k^2}
    \]
    para cualquier $k > 0$, o lo que es lo mismo
    \[
    P\{|X-E[X]| > c\} \leq \frac{\sigma^2}{c^2}
    \]
    para cualquier $c > 0$.
\end{proposition}

\begin{proposition}[Desigualdad de Markov]
    \[
    P\{f(X)>c\} \leq \frac{E[f(X)]}{c}
    \]
\end{proposition}

El momento central de tercer orden
\[
\mu_3 = \sum_k (x_k-\alpha_1)^3 p_k
\]
mide la asimetría de la distribución alrededor de su media. También se usa el \emph{coeficiente de asimetría de la distribución}.
\[
\gamma_3 = \frac{\mu_3}{\sigma_3}
\]

El \emph{coeficiente de apuntalamiento} o \emph{curtosis de la distribución}
\[
\gamma_4 = \frac{\mu_4}{\sigma_4} - 3
\]

\section{Momentos de una distribución conjunta}

Respecto al origen, los momentos de la distribución conjunta con el valor esperado del producto de dos potencias de las variables.
\[
\sigma_{r,s}=E[X_1^rX_2^s]
\]

Respecto al centro de gravedad $(E[X_1],E[X_2])$ de la distribución, debe restarse a cada variable su media.
\[
\mu_{r,s} = E[(X_1-E[X_1])^r(X_2-E[X_2])^s]
\]

El \emph{orden} de uno de estos momentos es el número $r+s$.

El momento más utilizado es el valor esperado del producto de las dos variables o \emph{covarianza} entre ellas
\[
\alpha_{1,1} = E[X_1X_2]
\]
y, más aún, la \emph{covarianza} entre ellas
\[
\Cov(X_1,X_2) = \mu_{1,1} = E[(X_1 - E[X_1])(X_2 - E[X_2])] = E[X_1X_2]-E[X_1]E[X_2]
\]

Dos variables aleatorias $X_1$ y $X_2$ independientes, tienen covarianza nula. Pero dos variables con covarianza nula, no son independientes.

Otra muy usada es el \emph{coeficiente de correlación}
\[
\rho(X_1,X_2) = \frac{\Cov(X_1,X_2)}{\sigma(X_1)\sigma(X_2)}
\]

Las variables cuyo coeficiente de correlación es nulo se denominan \emph{incorreladas}.

La \emph{matriz de covarianzas} de $X_1$ y $X_2$ es la matriz
\[
\Sigma(X_1,X_2) = \begin{pmatrix}
    \sigma^2(X_1) & \Cov(X_1,X_2) \\
    \Cov(X_1,X_2) & \sigma^2(X_2)
\end{pmatrix}
\]

Es una matriz simétrica semidefinida positiva y con determinante positivo, por lo que
\[
\Cov(X_1,X_2)^2 \leq \sigma^2(X_1)\sigma^2(X_2) \qquad -1 \leq \rho(X_1,X_2) \leq 1
\]

Si $\rho(X_1,X_2) = \pm 1$ es equivalente a que cada una de las variables sea función lineal de la otra.

La mejor previsión de $X_2$, mediante una función lineal de $X_1$, es la \emph{recta de regresión de $X_2$ sobre $X_1$}
\[
x_2 = a^*(x_1 - E[X_1]) + E[X_2] \qquad a^* = \frac{\Cov(X_1,X_2)}{\sigma^2(X_1)}
\]

Siendo la \emph{varianza residual de $X_2$}, después de hacer la regresión sobre $X_1$.
\[
\sigma^2(X_2)(1 - \rho^2(X_1,X_2))
\]

Si $X_1,X_2,\ldots,X_n$ son variables aleatorias independientes, se verifica
\[
\sigma^2(X_1 + X_2 + \cdots + X_n) = \sigma^2(X_1) + \sigma^2(X_2) + \cdots + \sigma^2(X_n)
\]

\section{Otros indicadores de posición y dispersión}

\subsection{Indicadores de posición}

\begin{description}
    \item[Moda] La \emph{moda} de una distribución que asigna probabilidades $p_1,p_2,\ldots$ a los puntos $x_1,x_2,\ldots$ es el valor $x_k$ para el cual $p_k$ es máximo.
    \item[Mediana]  La \emph{mediana} de una distribución es aquel valor $M$ que deja probabilidad $\frac{1}{2}$ tanto por debajo como por encima de él.
    \[P\{X \leq M\} \geq \frac{1}{2} \qquad P\{X \geq M\} \geq \frac{1}{2}\]
    Al igual que la media es el valor $a$ que minimiza $E[(X-a)^2]$, la mediana es el valor que minimiza.
    \[
    E[|X-a|]
    \]
\end{description}

\subsection{Indicadores de dispersión}

\begin{description}
    \item[Promedio de las desviaciones absolutas a $a$] Consiste en calcular $E[|X-a|]$ tomando $a$ como $E[X]$ o $M(X)$.
    \item[Desviación probable] Si $M(X)$ es única se puede calcular la \emph{mediana de las desviaciones absolutas a la mediana}
    \[M(|X-M(X|)\]
    Tiene la virtud que si su valor es $D$, entonces el intervalo $[M(X)-D,M(X)+D]$ tiene probabilidad superior a $\frac{1}{2}$. Y si el intervalo es $(M(X)-D,M(X)+D)$ tiene probabilidad inferior a $\frac{1}{2}$.
    \item[Cuantil de orden $p$] Dado cualquier número $p \in [0,1]$, el menor valor $x_k$ de la variable tal que $P\{X \leq x_k\} \geq p$ se denomina el \emph{cuantil de orden $p$} de la distribución.
    \[c_p = \min\{x_k | F(x_k) \geq p\}\]
    $c_{\frac{1}{2}}$ es la mediana. $c_{\frac{1}{4}}$ y $c_{\frac{3}{4}}$ el primer y tercer cuartil. Y el \emph{intervalo intercuartílico} $[c_{\frac{1}{4}}, c_{\frac{3}{4}}]$ tiene probabilidad igual o superior a $\frac{1}{2}$
\end{description}

\section{Función generatriz}

Si $X$ es una variable aleatoria con valores enteros no negativos, con $P\{X = n\} = p_n$ para cada $n$, se llama función generatriz de $X$ a la función
\[
G(z) = E[z^X] = \sum_{n=0}^{\infty} z^np_n
\]

Si $X_1$ y $X_2$ son dos variables aleatorias independientes, con función generatrices $G_1(z)$ y $G_2(z)$, la función generatriz de la suma $X_1 + X_2$ es el producto $G_1(z)G_2(z)$.


\chapter{Apéndice I. Combinatoria} 

\section{Principios generales} 

\textbf{Contar} es hallar el \textbf{cardinal} de un conjunto. La \textbf{combinatoria} es el arte de contar conjuntos sin hacer enumeraciones.

Si en una competición se inscriben \(n\) jugadores se juegan \(n-1\) partidos.

El \textbf{procedimiento constructivo} consiste en recorrer mentalmente los pasos a seguir para formar todos los elementos del conjunto,
anotando las alternativas que pueden elegirse en cada uno. 

Si los conjuntos \(A_1, A_2, \ldots, A_n\) tienen \(n_1, n_2,\ldots, n_n\) elementos respectivamente, el producto cartesiano \(A_1 \times A_2 \times \cdots \times A_n\) tiene
\(n_1n_2\cdots n_n\) elementos.

No tiene ninguna relevancia donde se producen las restricciones, solo cuales son estas.

\section{Patrones más usuales}
\label{patrones-muxe1s-usuales}

\subsection{Ordenaciones}
\label{ordenaciones}

\(n\) objetos distintos pueden ordenarse en fila de \(n!\) maneras distintas.

Una colección de \(n\) objetos, clasificados en \(k\) grupos de objetos idénticos entre sí, el primero con \(n_1\) objetos, el segundo con
\(n_2\),\ldots{} se pueden ordenar en fila de 
\[
\frac{n!}{n_1!n_2!\cdots n_k!}
\] 
maneras distintas, si no se consideran distintas las ordenaciones en las cuales dos objetos iguales han permutado su posición. 

\subsection{Subconjuntos ordenados} 

Hay \[n(n-1)(n-2)\cdots(n-r+1) = \frac{n!}{(n-r)!} = (n)_r\] subconjuntos ordenados posibles de \(r\) elementos, que pueden extraerse de un
conjunto de \(n\) elementos. También conocidos como \emph{variaciones sin repetición de \(r\) elementos tomados entre \(n\)}.

El número de \textbf{variaciones con repetición} de \(r\) elementos tomados entre \(n\) es \(n^r\). 

\subsection{Subconjuntos} 

El número de subconjuntos distintos, con \(r\) elementos, que pueden extraerse de un conjunto de \(n\) elementos es 
\[
\binom{n}{r}=\frac{n!}{r!(n-r!)}
\] 
o \textbf{combinaciones} de \(r\) elementos tomados entre \(n\).

La diferencia entre variaciones y subconjuntos es si el orden de aparición de los elementos es importante. 

\subsection{Repartos} 

Si hay que repartir \(r\) objetos iguales en \(n\) grupos, existen 
\[
\binom{n+r-1}{r}
\] 
repartos posibles. 

\section{Identidades combinatorias}

Una \textbf{identidad combinatoria} es una igualdad \(f(n)=g(n)\), válida para cada \(n\) natural, en la cual \(f(n)\) y \(g(n)\) son
cantidades relacionadas con el cardinal de algún conjunto. 
\[\binom{n}{k}=\binom{n}{n-k}\] \[2^n=\sum_{k=0}^n \binom{n}{k}\] \[
  \sum_{i=0}^n \binom{n}{i}^2 = \binom{2n}{n}
\] \[
  \sum_{k=0}^{m-1} \binom{n}{k}\binom{n-1}{m-k-1}=\binom{n+m-1}{n}
\] \[
  \binom{n}{k}=\binom{n-1}{k}+\binom{n-1}{k-1}
\] \[
  \binom{n}{k}=\sum_{j=0}^{k}\binom{n-j-1}{k-j}=\sum_{j=1}^{n-k+1}\binom{n-j}{k-1}
\] Si \(x \in \mathbb{R}\) y \(r >0\) \[
  \binom{x}{r} = \frac{x(x-1)(x-2)\cdots(x-r+1)}{r!}
\] y si \(r < 0\) \[
  \binom{x}{0} = 1 \qquad \binom{x}{r} = 0
\] \#\#\#\# Identidad inmediata \[
  \binom{-x}{r} = (-1)^r\binom{x+r-1}{r}
\] \#\#\#\# Desarrollo de Taylor Si \(|t|<1\) y para cualquier
\(x \in \mathbb{R}\) \[
  (1+t)^x=1 + \binom{x}{1}t + \binom{x}{2}t^2 + \cdots + \binom{x}{r}t^r+\cdots
\] y si \(|b| < |a|\) y cualquier \(x \in \mathbb{R}\) \[
  (a+b)^x=a^x + \binom{x}{1}ba^{x-1} + \binom{x}{2}b^2a^{x-2} + \cdots + \binom{x}{r}b^ra^{a-r}+\cdots
\]

\[
  (1-t)^n\sum_{r=0}^\infty \binom{n+r-1}{r}t^r=1
\] \[
  (-1)^n\binom{-\frac{1}{2}}{n}=\binom{2n}{n}\frac{1}{2^{2n}}
\] \[
  \sum_{n=0}^\infty \frac{2n}{n}\frac{t^n}{2^{2n}}=(1-t)^{\frac{-1}{2}}
\] y si \(n\) es un entero positivo \[
  \binom{x+y}{n}=\binom{x}{0}\binom{y}{n}+\binom{x}{1}\binom{y}{n-1}+\binom{x}{2}\binom{y}{n-2}+\cdots+\binom{x}{n}\binom{y}{0}
\] 

\section{Formula de Stirling} 
\[
  n! = \sqrt{2n}n^{n+\frac{1}{2}}e^{-n}
\] con más precisión \[
  1 < \frac{n!}{\sqrt{2n}n^{n+\frac{1}{2}}e^{-n}} < e^{\frac{1}{8n}}
\]

\chapter*{Desarrollo de series}

\begin{align}
	e^x &= \sum_{n=0}^\infty \frac{x^n}{n!}, \qquad x \in (-\infty,\infty) \\
	\sin x &= \sum_{n=0}^\infty (-1)^n \frac{x^{2n+1}}{(2n+1)!}, \qquad x \in (-\infty,\infty) \\ 
	\cos x &= \sum_{n=0}^\infty (-1)^n \frac{x^{2n}}{(2n)!}, \qquad x \in (-\infty,\infty) \\
	\frac{1}{1-x} &= \sum_{n=0}^\infty x^n, \qquad x \in (-1,1) \\
	\frac{1}{1+x} &= \sum_{n=0}^\infty (-1)^n x^n, \qquad x \in (-1,1) \\
	\log(1-x) &= \sum_{n=0}^\infty \frac{x^{n+1}}{n+1}, \qquad x \in (-1,1) \\
	\log(1+x) &= \sum_{n=0}^\infty (-1)^n \frac{x^{n+1}}{n+1}, \qquad x \in (-1,1) \\
	\log(\frac{1-x}{1+x}) &= 2\sum_{n=0}^\infty \frac{x^{2n+1}}{2n+1}, \qquad x \in (-1,1) \\
	\frac{1}{1+x*2} &= \sum_{n=0}^\infty (-1)^n x^{2n}, \qquad x \in (-1,1) \\
	\arctan(x) &= \sum_{n=0}^\infty (-1)^n \frac{x^{2n+1}}{2n+1}, \qquad x \in (-1,1) \\
	(1+x)^\alpha &= 1 + \sum_{n=1}^\infty \frac{\alpha(\alpha-1)\cdots(\alpha-n+1)}{n!}x^n, \qquad x \in (-1,1) \\
	\frac{1}{\sqrt{1+x}} &= 1 + \sum_{n=1}^\infty (-1)^n \frac{(2n-1)!!}{(2n)!!}x^n, \qquad x \in (-1,1) \\
	\frac{1}{\sqrt{1-x^2}} &= 1 +  \sum_{n=1}^\infty \frac{(2n-1)!!}{(2n)!!}x^{2n}, \qquad x \in (-1,1) \\
	\frac{1}{\sqrt{1+x^2}} &= 1 + \sum_{n=1}^\infty (-1)^n \frac{(2n-1)!!}{(2n)!!}x^{2n}, \qquad x \in (-1,1) \\
	\arcsin x &= x+ \sum_{n=1}^\infty  \frac{(2n-1)!!}{(2n)!!}\frac{x^{2n+1}}{2n+1}, \qquad x \in (-1,1) \\
	argsh x &= x+ \sum_{n=1}^\infty  (-1)^n \frac{(2n-1)!!}{(2n)!!}\frac{x^{2n+1}}{2n+1}, \qquad x \in (-1,1) 
\end{align}

\end{document}
